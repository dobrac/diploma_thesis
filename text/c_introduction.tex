%The aim of the work is to design and implement a static web presentation generator for gRPC API documentation. The input files are proto files that carry a description of services, calls and types. The output is an HTML page with documentation and the ability to call the API (provided that the API supports gRPC-web).
%1. Research existing solutions for Protocol Buffers and compare features with similar tools for GraphQL or RESTful APIs.
%2. Design and implement a documentation generator with the ability to call an existing gRPC-web API.
%3. Discuss and possibly implement an alternative data source using gRPC reflection.
%4. Test the application with automated tests and perform user testing.


%Having a client implementation and documentation for a server API helps with testing and development.
%But implementing the client takes time and effort.
%For this reason there are generic clients used for that.
%In the RESTful API world, Swagger is a popular choice.
%In the GraphQL world, popular might be GraphiQL\@.
%Therefore, it is desired to have a similar tool for gRPC\@.
%
%The tools for gRPC exists, but they lack user-friendliness compared the ones for REST or GraphQL\@.
%They require separate server to be run, and they don't always support documenting of services, calls and types.
%
%The aim of the work is to design and implement a static web presentation generator for gRPC API documentation.
%The input files are Protobuf Buffer files that carry a description of services, calls and types.
%The output is a static HTML page with documentation and the ability to call the API (provided that the API supports gRPC-web).
%
%I've chosen this topic because I believe I can offer a solution that will solve the troubles having documentation for gRPC services.
%This applies especially for developers who are creating mainly RESTful APIs\@.
%
%The thesis is split into four main parts.
%The first part is the analysis of existing solutions.
%The following parts are design, implementation and testing.
%
%
%In the first part, I analyze the features in existing solutions for gRPC and compare them with similar tools for RESTful and GraphQL APIs.
%Based on the gained knowledge, I define the requirements and use cases for the website.
%
%In the design chapter, I present a new solution using a static web page and tools for conversion between gRPC API definition and the web page understood format.
%I also discuss the options and limitations of using gRPC vs gRPC-web.
%Finally, I discuss and design the possibility of getting the gRPC API definition using the gRPC reflection.
%
%In the implementation chapter, I discuss the choice of technology, the implementation of the designed solution and the architecture of the application.
%I also discuss the problems that occurred during the implementation and their solutions.
%The chapter ends with a discussion of the licenses of the used libraries.
%
%The testing chapter is about manual and automated testing.
%The automated testing is done using unit tests.
%The manual testing is done using user testing, which focuses on the user-friendliness of the application.

Having a readily available client implementation and clear documentation for a server API significantly aids in testing and development.
However, implementing clients can be a time-consuming effort.
To address this, generic clients are often employed.
Swagger\footnote{\url{https://swagger.io/}} is a popular choice for RESTful APIs, while GraphiQL\footnote{\url{https://github.com/graphql/graphiql}} often serves this purpose in the GraphQL world.
There is an evident need for a similarly streamlined tool within the gRPC ecosystem.

While gRPC documentation tools exist, they often lack the user-friendliness found in their REST or GraphQL counterparts.
These solutions may require separate servers and sometimes have limited support for comprehensively documenting services, calls, and types.

The central aim of this work is to design and implement a static web presentation generator focused on user-friendly gRPC API documentation.
Input will consist of Protobuf Buffer files that describe the services, calls, and types.
The desired output is a static HTML page that provides clear documentation and the ability to directly execute gRPC-web API calls (assuming the API supports gRPC-web).

I have chosen this topic because I believe I can offer a solution that will alleviate the challenges associated with documenting gRPC services.
This is particularly relevant for developers who primarily work with RESTful APIs.

The thesis is divided into four main parts.
The first part focuses on analyzing existing solutions.
Subsequent parts cover design, implementation, and testing.

In the analysis phase, I will examine the features present in existing gRPC documentation solutions and compare them to similar tools designed for RESTful and GraphQL APIs.
Based on the insights gained, I will define the specific requirements and use cases for the static web presentation generator.

In the design chapter, I will present a solution that leverages a static web page and tools for converting gRPC API definitions into a format readily understood by the web page.
I will also delve into the options and potential limitations of using gRPC vs gRPC-web.
Additionally, I will explore and design a mechanism to obtain a gRPC API definition using gRPC reflection.

The implementation chapter will address the selection of an appropriate technology stack, the implementation of the designed solution, and the overall architecture of the application.
I will discuss any challenges encountered during implementation and the solutions devised.
This chapter will conclude with a review of the licenses associated with any external libraries used.

Finally, the testing chapter will cover both manual and automated testing strategies.
Automated testing will be conducted utilizing unit tests.
Manual testing will be performed through user testing, with a strong emphasis on evaluating the user-friendliness of the application.