For the website testing, I have prepared automated unit tests and manual scenarios.
The automated tests are written in the TypeScript programming language using the Jest~\footnote{\url{https://jestjs.io/}} framework.
The manual scenarios serve as a guide for the user testing of the website
and for verifying the correctness of the website's behavior.


\section{Automated Testing}
Using the Jest framework, I have created unit tests for the most critical parts of the website.
The tests cover the services and components of the website.
They are located in the \texttt{\_\_tests\_\_} directory.
The test coverage statistics are in the table~\ref{tab:tests-coverage}.

\begin{table}[hbt!]
    \centering
    \captionsetup{justification=centering}
    \begin{tabular}{|l|l|l|l|l|}
        \hline
        ~        & Statements & Branches & Functions & Lines   \\
        \hline
        Coverage & 97.46\%    & 95.73\%  & 86.74\%   & 97.46\% \\
        \hline
    \end{tabular}
    \caption{Test coverage statistics}
    \label{tab:tests-coverage}
\end{table}


\section{Manual Scenarios}
For the manual user testing of the website, I have created scenarios that cover all use cases.
Each scenario consists of:
\begin{itemize}
    \item initial state,
    \item steps,
    \item covered use cases.
\end{itemize}

In the individual steps, I describe how to achieve the result defined in the title of the testing scenario.

\newcounter{testingcounter}
\newcommand{\testing}[1]{%
    \stepcounter{testingcounter}%
    \subsection{T\arabic{testingcounter} -- #1}
}


\testing{Generating the website from proto files and validating the data}
Initial state: A folder with proto files\\
Steps:
\begin{enumerate}
    \item Open the terminal
    \item Write a command for translating proto files to JSON \textit{gf-proto-to-json \. > common.json}
    \item Open the website (locally or hosted), change the input to file, and upload the JSON file
    \item Check if the data is correctly displayed, including the services, methods, and messages
    \item Check if the comments and options are present
\end{enumerate}
Use cases covered: UC1, UC3, UC4, UC5, UC6, UC7

\testing{Generating the website from the gRPC reflection and validating the data}
Initial state: A gRPC server URL with reflection enabled\\
Steps:
\begin{enumerate}
    \item Open the terminal
    \item Write a command for getting the bin file from the gRPC server \textit{grpcurl -protoset-out descriptors.bin -plaintext \$\{GRPC\_SERVER\_URL\} describe}
    \item Write a command for translating the bin descriptors file to JSON \textit{gf-reflection-to-json descriptors.bin > common.json}
    \item Open the website (locally or hosted), change the input to file, and upload the bin file
    \item Check if the data is correctly displayed, including the services, methods, and messages
    \item Check if the comments and options are present
\end{enumerate}
Use cases covered: UC2, UC3, UC4, UC5, UC6, UC7

\testing{Executing unary request}
Initial state: Hosted website with loaded definitions and gRPC server running with the Envoy proxy\\
Steps:
\begin{enumerate}
    \item Open the website with definitions loaded
    \item Open the unary request
    \item Fill in the required fields
    \item Click on the \enquote{Execute} button
    \item Check if the response is displayed with headers and trailers
\end{enumerate}
Use cases covered: UC8

\testing{Executing server streaming request}
Initial state: Hosted website with loaded definitions and gRPC server running with the Envoy proxy\\
Steps:
\begin{enumerate}
    \item Open the website with definitions loaded
    \item Open the server streaming request
    \item Fill in the required fields
    \item Click on the \enquote{Execute} button
    \item Check if the responses are being displayed as they arrive
    \item Check if the headers and trailers are displayed
\end{enumerate}
Use cases covered: UC9

\testing{Setting global metadata}
Initial state: Hosted website with loaded definitions and gRPC server running with the Envoy proxy\\
Steps:
\begin{enumerate}
    \item Open the website with definitions loaded
    \item Click the metadata button
    \item Set any key-value metadata pair
    \item Set the authorization token
    \item Close the metadata dialog
    \item Open any unary request
    \item Fill in the required fields
    \item Click on the \enquote{Execute} button
    \item Check if the request contains the set metadata
\end{enumerate}
Use cases covered: UC10


\section{User Testing}
My goal for user testing is to determine the intuitiveness and clarity of the application and to identify any issues.
For the user testing, I have chosen developers.

I have created the following scenarios for the user testing:
\begin{itemize}
    \item Common Format Generation from Proto Files
    \item List Services, Methods, Message Types, and Enum Types
    \item Comments and Options
    \item Execute Unary Request
    \item Execute Server Streaming Request
    \item Complex Method Input
    \item Global Metadata
\end{itemize}

The scenarios are different from those for testing the functionality of the application
and do not cover the entire functionality
because the primary goal of user testing is the intuitiveness and clarity of the application.

Each user will be briefed about the application's general use-case,
asked about their experience with similar pieces of software,
and then they will be asked to complete the scenarios
and provide feedback on the application's functionality and clarity.

The pre-questionnaire will be used to gather information about the user's experience with similar software.
The questions are:
\begin{itemize}
    \item What is your main focus in programming?
    (software, hardware, web development, etc.)
    \item What is your experience with REST APIs?
    \item What is your experience with Swagger UI, GraphiQL, or similar tools?
    \item What is your experience with the gRPC?
\end{itemize}

The post-questionnaire will be used to gather feedback on the application's functionality and clarity.
The questions are:
\begin{itemize}
    \item How did you find the common format generation process?
    (rate 1--5, 1 being the best)
    \item How did you find the overall application design?
    (rate 1--5, 1 being the best)
    \item How difficult was to find the methods and message types?
    (rate 1--5, 1 being the easiest)
    \item How difficult was to control the method execution?
    (rate 1--5, 1 being the easiest)
    \item Is there anything else to add?
\end{itemize}

\subsection{Common Format Generation from Proto Files}
The tester is provided with proto files and a script for generating a common format with its documentation.

\textbf{The following instructions are given:}\\
Based on the provided proto files, generate a common format which will be used later on for the documentation website.

\textbf{The expected result is:}\\
A generated common format JSON file.

\subsection{List Services, Methods, Message Types, and Enum Types}
The tester is provided with the common format file.
The application is running on localhost on port 3000.

\textbf{The following instructions are given:}\\
You have a website running on the localhost on port 3000.
List all services, methods, message types, and enum types from the common format file.
Find the method \texttt{SayHello} and tell me what type it gives as a response.

\textbf{The expected result is:}\\
The method is found and the type is \texttt{HelloReply}.

\subsection{Comments and Options}
The tester is provided with the common format file.
The application is running on localhost on port 3000.

\textbf{The following instructions are given:}\\
You have a website running on the localhost on port 3000.
What does the method \texttt{ListFeatures} in the service \texttt{RouteGuide} do, and what is the java package of that service?

\textbf{The expected result is:}\\
The method obtains features within the given rectangle and the java package is \textit{guiding.route}.

\subsection{Execute Unary Request}
The tester is provided with the common format file.
The application is running on localhost on port 3000, and the gRPC-Web proxy with gRPC server is set up.

\textbf{The following instructions are given:}\\
You have a website running on the localhost on port 3000.
Execute method \texttt{SayHello} with their parameters and tell me the response with their headers and trailers.

\textbf{The expected result is:}\\
The method is executed with any parameter
and the response is \enquote{Hello Name} in the \texttt{message} field of the response message type.

\subsection{Execute Server Streaming Request}
The tester is provided with the common format file.
The application is running on localhost on port 3000, and the gRPC-Web proxy with gRPC server is set up.

\textbf{The following instructions are given:}\\
You have a website running on the localhost on port 3000.
Execute method \texttt{SayRepeatHello} with their parameters and tell me the responses with their headers and trailers.

\textbf{The expected result is:}\\
The method is executed with any parameter
and the response is \enquote{Hello Name} multiple times in the \texttt{message} field of the response message type.

\subsection{Complex Method Input}
The tester is provided with the common format file.
The application is running on localhost on port 3000, and the gRPC-Web proxy with gRPC server is set up.

\textbf{The following instructions are given:}\\
You have a website running on the localhost on port 3000.
Execute method \texttt{ListFeatures} with their parameters and tell me the responses.
The parameters are the following:
\begin{itemize}
    \item \texttt{lo}: latitude: 400000000, longitude: -750000000,
    \item \texttt{hi}: latitude 420000000, longitude -73000000.
\end{itemize}

\textbf{The expected result is:}\\
The method is executed with and responses are returned.

\subsection{Global Metadata}
The tester is provided with the common format file and authorization token.
The application is running on localhost on port 3000, and the gRPC-Web proxy with gRPC server is set up.

\textbf{The following instructions are given:}\\
You have a website running on the localhost on port 3000.
Execute method \texttt{SayHello} with the authorization token.

\textbf{The expected result is:}\\
The method is executed, and in the request metadata section, the authorization token is present.

\subsection{Found Issues and Their Solutions}\label{subsec:found-issues-and-their-solutions}
%\begin{itemize}
%    \item Pro vyřešení problému s nepřehledností vybraného semestru jsem do dialogového okna s informacemi o inicializaci semestru přidal stejný výběr semestru, jako v hlavním okně aplikace.
%    \item Při převodu časových lístků (včetně rozvrhu) nepovoluji vznik kolizí, i když to v softwaru Syllabus Plus lze.
%    Nicméně bylo mi řečeno, že je lepší, pokud se vyhodí výjimka (tak jako nyní), než aby program ignoroval omezení a potom vznikly nechtěné kolize.
%    \item Tlačítka pro aktualizaci dat jednotlivých systémů byla hledána spíše v horní části programu, proto jsem je přesunul pod tlačítko s načtením všech dat.
%    \item Pro zpřehlednění původu časového lístku jsem přidal do dialogového okna další text s informací o systému (nenachází se nadále pouze v hlavičce).
%    \item Během testování bylo lehké zmatení mezi označenými položkami pro převod a otevřenými položkami pro zobrazení ve sloupcích.
%    Dle zpětné vazby není jasné, kolik položek je označených.
%    Jako řešení jsem navrhl přesunout informaci o počtu vybraných položek mimo tlačítko pro migraci.
%    Změna je zachycena na obrázku~\ref{fig:ui-migrate-diff}.
%    \item Během testování vyhledání učitele bylo zjištěno zmatení s prázdným sloupcem s učiteli při nevybrání kategorie.
%    Navrhl jsem proto řešení, které v případě nevybrané kategorie sloupec znepřístupní a zobrazí text s instrukcí pro jeho aktivaci (pokyn k vybrání položky v nadřazené kategorii).
%    Změna je zachycena na obrázku~\ref{fig:ui-column-diff}.
%    \item Akce sbalení sloupce není jednoduše dohledatelná, nicméně při nahodilém klikání na ni přišel.
%    Jako řešení jsem navrhl podtrhávání textu názvu sloupce při najetí.
%\end{itemize}
%
%\bigskip
%
%Další poznámky, které se netýkaly přímo testovacích scénářů, ale v rámci práce na ně bylo poukázáno:
%\begin{itemize}
%    \item zobrazení tlačítka pro vybrání všech položek ve sloupci, i když v daném sloupci lze vybrat maximálně jednu položku,
%    \item chybějící zobrazení místností a učitelů, kteří nejsou přiřazeni do žádné skupiny,
%    \item zaškrtávací políčko pro schování předmětů s prázdnými časovými lístky bylo v menší velikosti okna zablokováno jinými prvky (změna zaškrtávacího políčka je zachycena na obrázku~\ref{fig:ui-column-diff}).
%\end{itemize}

\subsection{Testing Summary}
%Všechny akce uživatelského testování kromě zjištěných problémů byly na základě zpětné vazby intuitivní a dostatečně přehledné.
%Dále bylo během převodu oceněno okamžité propsání změn v softwaru Syllabus Plus.
%
%Testování ukázalo, že výsledná aplikace je přehledná, funkční v prostředí, ve kterém bude software používaný, a neobsahuje chyby ovlivňující funkcionalitu.
%Po opravení zjištěných problémů (viz podkapitola~\ref{subsec:zjistene-problemy-a-jejich-reseni}) je intuitivnost aplikace splněna.
%
%Jako nápad na vylepšení byla zmíněna možnost zobrazení detailu časových lístků pro více předmětů najednou a možnost výběru více položek pomocí klávesové zkratky \enquote{ctrl + shift}.
%Na základě krátké konverzace ale nebylo zřejmé, zda by to mělo reálné použití v tvorbě rozvrhu, proto bych nechal tento nápad spíše do případného vylepšení aplikace.
%
%Manuální scénáře potvrdily splnění a funkčnost všech požadavků aplikace.
%Otestování správného výpisu jednotlivých položek s exportem časových lístků bylo provedeno odděleně s několika ukázkovými záznamy, kdy exportovaná data měla správný formát a patřičné změny se korektně provedly.
%



\section{Testing Summary}
The user testing feedback was positive, and the testers found the website intuitive and clear.
It was also appreciated that the process and documentation presentation are familiar to the Swagger UI\@.
The generation of the common format was simple, but it was suggested to have a button for the generation in the UI\@.
The generation is designed to be performed by automatic deployment or other tools, so having the UI would change its purpose.
Therefore, it is not planned to be implemented.

The user testing showed that the resulting website fulfills the documentation purpose and allows the method calls, and does not contain any functionality-affecting bugs.
After fixing the found usability issues (see subsection~\ref{subsec:found-issues-and-their-solutions}), the intuitiveness of the website is met.

The automatic and manual testing confirmed the fulfillment and functionality of all the static website generator requirements.