%Hlavním cílem je vytvořit aplikaci s grafickým uživatelským rozhraním, která zjednoduší a zefektivní převody dat mezi softwarem Syllabus Plus a systémem KOS pro fakultního rozvrháře.
%Toho se snaží dosáhnout pomocí vizualizace dat, zobrazení rozdílů, možnosti částečných převodů a samotným grafickým uživatelským rozhraním.
%
%Cílem analýzy je zmapovat proces tvorby rozvrhu na FITu, jeho problémy a požadavky.
%Na základě zjištěných informací navrhnout řešení, které analyzované požadavky splní, a umožní fakultnímu rozvrháři převody dat mezi systémy v obou směrech.
%
%V realizaci je cílem vytvoření grafické aplikace.
%Ta bude umět vizualizovat data jednotlivých systémů a provádět kompletní nebo částečné převody.
%
%V kapitole testování je cílem ověřit funkčnost výsledné aplikace a provést uživatelské testování zaměřené na intuitivnost a přehlednost uživatelského rozhraní.
%
%Výsledná práce bude přínosná jak pro aktuálního fakultního rozvrháře, tak i pro budoucí tím, že zjednoduší a zpřehlední převody potřebných dat mezi systémy a odstraní aktuální problémy.