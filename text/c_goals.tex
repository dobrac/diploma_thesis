%The main goal is to create a static HTML page with documentation and the ability to call the gRPC API (provided that the API supports gRPC-web).
%The input files are Protobuf Buffer files that carry a description of services, calls and types.
%It tries to solve the problem using a static web page and tools for conversion between gRPC API definition and the web page understood format.
%
%The goal of the analysis is to research the existing solutions for gRPC and compare their features with similar tools for RESTful and GraphQL APIs.
%The gained knowledge is then used to define the requirements and use cases for the website, and to design a new solution using a static web page and tools for conversion between gRPC API definition and the web page understood format.
%
%In the implementation chapter, my goal is to create the static HTML page with the ability to call the gRPC API\@.
%The website will also support the creating the documentation from gRPC reflection.
%
%In the testing chapter, my goal is to test the application using automated tests and perform user testing.
%The user testing will focus on the user-friendliness of the application.
%
%The resulting work will be beneficial for developers who are creating gRPC APIs, as it will provide a solution that will solve the troubles having interactive documentation for gRPC services.

The primary goal of this thesis is to develop a static HTML page that provides both developer-familiar documentation design and gRPC API interaction capabilities (assuming the API supports gRPC-web).
This project will leverage Protocol Buffer files, which define services, calls, and data types as input.
A key focus is to address the challenges of creating interactive gRPC API documentation by utilizing a static web page and tools facilitating conversion between the gRPC API definition and a format readily understood by web pages.

A comprehensive analysis of existing gRPC solutions, alongside comparable tools for RESTful and GraphQL APIs, will be conducted.
This analysis will examine features, strengths, and weaknesses, informing the specific requirements and use cases for the proposed web-based solution.

The implementation chapter centers on the creation of the static HTML page, enabling direct gRPC API calls.
Additionally, this page will incorporate the ability to automatically generate documentation leveraging gRPC reflection.

A testing phase will include both automated tests to verify functionality and user testing to evaluate the interface's overall usability and developer-familiar design.

The ultimate outcome of this work is to provide a valuable resource for developers building or using gRPC APIs.
This solution will streamline the process of creating interactive documentation, enhancing understanding and efficient use of gRPC services, and hosting the documentation website.