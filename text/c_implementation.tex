% Talk about issues encountered, how they were solved
In this section, I will describe the implementation of the website generator.
I will start by discussing the choice of technology, followed by the project settings.
Then, I will describe the code structure, user interface, and functionality.
Finally, I will discuss the licensing of the libraries used in the project.


\section{Choosing the Technology}
The main factor in choosing the technology was the ability to generate a website that can be easily deployed and accessed by users.
In the design chapter, I have concluded that I will use a common format with all gRPC definitions, and the website will be generated from this format in the browser.
Therefore, I need to choose a technology for the website and for the common format generators.

As of the website, the basics are done using HTML and CSS\@.
There is no other choice.
For the programming language, because of the dynamic rendering based on the common format, the only possibilities are JavaScript and WebAssembly.
Because I will need to be updating the website (meaning the DOM), which is not supported directly by WebAssembly, I have chosen JavaScript~\cite{webassembly-dom}.

Because JavaScript is a weakly typed language, which, in my experience, can lead to bugs in larger projects, there are supersets and other libraries trying to add strong typing.
The popular libraries that I know of are TypeScript and Flow.
Because I have the most experience with TypeScript, I will be using it instead of direct JavaScript.

For the common format generators, I have many options because the generators are run on the developer's machine.
Therefore, the decision is based on the libraries needed for the generators.
And, because I will be using the protobufjs JavaScript library for the common format, I will use JavaScript for the generators.

\subsection{Web Framework}
In order to create the website with extensive logic, I will use a front-end framework.
There are several options, but the most popular are React, Angular, Vue.js, and Svelte~\cite{state-of-js-frontend-frameworks}.
I want the page to exist for a long time and be maintained.
For this reason, I will choose the most popular framework, which is React.

React is a JavaScript library for building user interfaces.
It is maintained by Meta and a community of individual developers and companies.
React can be used as a base in the development of single-page web applications.
It allows developers to create large web applications that can change data without reloading the page, making the website faster.
\cite{react}

Using React is a great option, but it requires a lot of parts to be set up manually.
For this reason, there is a React framework called Next.js.
It simplifies the setup, development, static site page generation, routing, and a lot more~\cite{nextjs}.
It is also the first recommended way to build a new React application by the React team~\cite{react-start-new-project}.
Therefore, I will use Next.js for the website.
The key features except the setup of Next.js that I will use are static site generation and TypeScript support.

\subsection{Styling Libraries}

\subsection{protobufjs Library}
% and protobufjs-cli

\subsection{gRPC-web Client Library}
% Talk about extracting gRPC-Web client from the generated stubs

\subsection{Other Libraries}


\section{Project Settings}
% lerna, pnpm


\section{JSON from Proto Files Generator}


\section{JSON from gRPC Reflection Generator}


\section{User Interface}

\subsection{Building}


\section{User Interface}


\section{Licensing}
For the development, I used libraries that use the following licenses:
\begin{itemize}
    \item MIT\footnote{\url{https://choosealicense.com/licenses/mit/}},
    \item BSD-3-Clause\footnote{\url{https://opensource.org/license/bsd-3-clause}},
    \item Apache 2.0\footnote{\url{https://choosealicense.com/licenses/apache-2.0/}},
    \item ISC\footnote{\url{https://www.isc.org/licenses/}},
    \item CC-BY-4.0\footnote{\url{https://creativecommons.org/licenses/by/4.0/}}.
\end{itemize}

The list of libraries and their licenses is captured in tables~\ref{tab:libraries-licenses} and~\ref{tab:libraries-licenses-dev}.
The rights and limitations of these licenses are then shown in table~\ref{tab:licenses}.

\begin{table}
    \centering
    \captionsetup{justification=centering}
    \begin{tabular}{|l|l|l|l|l|l|}
        \hline
        & \textbf{MIT} & \textbf{BSD-3-Clause} & \textbf{Apache 2.0} & \textbf{CC-BY-4.0} & \textbf{ISC} \\ \hline
        \textbf{Permissions}         &              &                       &                     &                    &              \\ \hline
        Commercial use               & \checkmark   & \checkmark            & \checkmark          & \checkmark         & \checkmark   \\ \hline
        Modification                 & \checkmark   & \checkmark            & \checkmark          & \checkmark         & \checkmark   \\ \hline
        Distribution                 & \checkmark   & \checkmark            & \checkmark          & \checkmark         & \checkmark   \\ \hline
        Patent use                   & -            & -                     & \checkmark          & x                  & -            \\ \hline
        Private use                  & \checkmark   & \checkmark            & \checkmark          & \checkmark         & \checkmark   \\ \hline
        &              &                       &                     &                    &              \\ \hline
        \textbf{Conditions}          &              &                       &                     &                    &              \\ \hline
        License and copyright notice & \checkmark   & \checkmark            & \checkmark          & \checkmark         & \checkmark   \\ \hline
        State changes                & -            & -                     & \checkmark          & \checkmark         & -            \\ \hline
        &              &                       &                     &                    &              \\ \hline
        \textbf{Limitations}         &              &                       &                     &                    &              \\ \hline
        Trademark use                & -            & -                     & x                   & x                  & -            \\ \hline
        Liability                    & x            & x                     & x                   & x                  & \checkmark   \\ \hline
        Warranty                     & x            & x                     & x                   & x                  & \checkmark   \\ \hline
    \end{tabular}
    \caption{Overview of licenses and their limitations}
    \label{tab:licenses}
\end{table}

All licenses allow both private and commercial use, including distribution and possible modifications.
The only requirement is to keep the license and copyright notice, eventually state changes if made.
% TODO
%All used libraries with their respective licenses can be found in the website


Because I have met the requirements of the licenses and their limitations allow me to use the libraries for free, I can use the libraries for my work.

\newcommand{\library}[1]{%
    #1\tablefootnote{\url{https://www.npmjs.com/package/#1}}%
}

\newpage
\begin{table}[hbt!]
    \centering
    \captionsetup{justification=centering}
    \begin{tabular}{|l|l|l|}
        \hline
        \textbf{Library}                              & \textbf{License} \\ \hline
        \library{@fortawesome/fontawesome-svg-core}   & MIT              \\ \hline
        \library{@fortawesome/free-regular-svg-icons} & CC-BY-4.0, MIT   \\ \hline
        \library{@fortawesome/free-solid-svg-icons}   & CC-BY-4.0, MIT   \\ \hline
        \library{@fortawesome/react-fontawesome}      & MIT              \\ \hline
        \library{@hookform/resolvers}                 & MIT              \\ \hline
        \library{bootstrap}                           & MIT              \\ \hline
        \library{grpc-web}                            & Apache-2.0       \\ \hline
        \library{lodash}                              & MIT              \\ \hline
        \library{next}                                & MIT              \\ \hline
        \library{p-cancelable}                        & MIT              \\ \hline
        \library{protobufjs}                          & BSD-3-Clause     \\ \hline
        \library{protobufjs-cli}                      & BSD-3-Clause     \\ \hline
        \library{react}                               & MIT              \\ \hline
        \library{react-bootstrap}                     & MIT              \\ \hline
        \library{react-dom}                           & MIT              \\ \hline
        \library{react-hook-form}                     & MIT              \\ \hline
        \library{react-syntax-highlighter}            & MIT              \\ \hline
        \library{sass}                                & MIT              \\ \hline
        \library{yup}                                 & MIT              \\ \hline
    \end{tabular}
    \caption{List of libraries and their licenses}
    \label{tab:libraries-licenses}
\end{table}

\newpage
\begin{table}[hbt!]
    \centering
    \captionsetup{justification=centering}
    \begin{tabular}{|l|l|l|}
        \hline
        \textbf{Development Library}              & \textbf{License} \\ \hline
        \library{@testing-library/jest-dom}       & MIT              \\ \hline
        \library{@testing-library/react}          & MIT              \\ \hline
        \library{@types/jest}                     & MIT              \\ \hline
        \library{@types/lodash}                   & MIT              \\ \hline
        \library{@types/node}                     & MIT              \\ \hline
        \library{@types/react}                    & MIT              \\ \hline
        \library{@types/react-dom}                & MIT              \\ \hline
        \library{@types/react-syntax-highlighter} & MIT              \\ \hline
        \library{cross-env}                       & MIT              \\ \hline
        \library{eslint}                          & MIT              \\ \hline
        \library{eslint-config-next}              & MIT              \\ \hline
        \library{eslint-config-prettier}          & MIT              \\ \hline
        \library{jest}                            & MIT              \\ \hline
        \library{jest-environment-jsdom}          & MIT              \\ \hline
        \library{prettier}                        & MIT              \\ \hline
        \library{typescript}                      & Apache-2.0       \\ \hline
        \library{lerna}                           & MIT              \\ \hline
        \library{rimraf}                          & ISC              \\ \hline
    \end{tabular}
    \caption{List of development libraries and their licenses}
    \label{tab:libraries-licenses-dev}
\end{table}