This tool helps you interact with gRPC services.
You can use it to explore the service's endpoints and make requests to them, browse types and enums, and preview options.


\section{Prerequisites}
To run this application, you need to have the libraries installed.

\begin{lstlisting}[language=bash, label={lst:documentation-install}]
pnpm install
\end{lstlisting}


\section{Usage}
Here are the commands you can use to run the application or generate JSON common format for the proto files.

\subsection{Website}
Information about the website compilation and development or production build.

\subsubsection{Development}
To run the website in development mode, use the following command.

\begin{lstlisting}[language=bash, label={lst:documentation-web-dev}]
pnpm -C web run dev
\end{lstlisting}

\subsubsection{Production}
To run the website in production mode, use the following commands.
First, build the website and then start the server.
The server will be available at \url{http://localhost:3000} by default.

\begin{lstlisting}[language=bash, label={lst:documentation-web-build}]
pnpm -C web run build
pnpm -C web run start
\end{lstlisting}

\subsection{Proto Files to JSON Generation}
Generates a JSON from the proto files.
The source files can be a single file or a list of files separated by a space or a folder/folders.

\begin{lstlisting}[language=bash, label={lst:documentation-generator-proto}]
gf-proto-to-json ${SOURCE_PROTO_FILES} > ${EXPORTED_NAME}.json
\end{lstlisting}

\subsection{Reflection to JSON Generation}
Generates a JSON from the reflection.
The source file should be a single file in the .bin format.

In the following example, the source file is a protoset file.
The protoset file can be created using the grpcurl\footnote{\url{https://github.com/fullstorydev/grpcurl}} tool.

\begin{enumerate}
    \item Create a protoset file (it can be done using different ways, this is just an example).
    \begin{lstlisting}[language=bash, label={lst:documentation-generator-reflection-grpcurl}, breaklines=true]
grpcurl -protoset-out descriptors.bin -plaintext localhost:8980 describe
    \end{lstlisting}


    \item Generate a JSON from the protoset file.
    The file should be a single file in the .bin format.

    \begin{lstlisting}[language=bash, label={lst:documentation-generator-reflection-bin}]
gf-reflection-to-json ${SOURCE_BIN_FILE} > ${EXPORTED_NAME}.json
    \end{lstlisting}


\end{enumerate}


\section{Testing Server}
To test the application, you can use the example testing server and Envoy proxy.
It is a simple gRPC server that has a few endpoints and types.

(Source of the Node.js server and Envoy proxy configuration:~\url{https://github.com/grpc/grpc-web/tree/master/net/grpc/gateway/examples/helloworld})

\begin{enumerate}
    \item Go to the example folder.
    \begin{lstlisting}[language=bash, label={lst:documentation-testing-cd-example}]
cd example
    \end{lstlisting}

    \item Start the Envoy proxy.
    (Linux users: Use address: localhost instead of address: host.docker.internal in the bottom section.)

    \begin{lstlisting}[language=bash, label={lst:documentation-testing-envoy}, breaklines=true]
docker run -d -v "$(pwd)"/envoy-proxy.yaml:/etc/envoy/envoy.yaml:ro -p 8080:8080 -p 9901:9901 envoyproxy/envoy:v1.22.0
    \end{lstlisting}

    \item Run the gRPC server.
    \begin{lstlisting}[language=bash, label={lst:documentation-testing-node}]
node server.js
    \end{lstlisting}

    \item Access the (already running) website and set the server URL to~\url{http://localhost:8080}.

\end{enumerate}