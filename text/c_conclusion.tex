%V práci jsem se zabýval analýzou procesu tvorby rozvrhu, návrhem, implementací a testováním aplikace.
%
%V rámci analýzy jsem zjistil, jak funguje proces tvorby rozvrhu na Fakultě informačních technologií.
%V procesu jsem se potom zaměřil na část s převody dat mezi systémem KOS a softwarem Syllabus Plus.
%Zjistil jsem, jak funguje aktuální řešení a analyzoval jeho problémy.
%Těmi hlavními jsou částečné opakované převody, přehlednost operací pro fakultního rozvrháře a uživatelské rozhraní.
%Na základě zjištěných informací jsem pak sestavil požadavky a případy užití.
%
%V další kapitole jsem se zabýval návrhem řešení.
%Během návrhu jsem přišel na to, že nejlepší možností je mít všechna data v jedné aplikaci a umožnit uživateli rozhodnout, co s daty udělá, na základě vizualizace se zobrazením rozdílů položek jednotlivých systémů.
%Uživatel tak může identifikovat případné další nekonzistence, které mohly při práci vzniknout, a předcházet zbytečným chybám.
%Díky tomu také odpadá nutnost poznamenávání úprav stranou při doplnění nových informací do systému KOS\@.
%Povedlo se mi ale také vylepšit připojení k softwaru Syllabus Plus.
%Místo využití pomocné databáze se má aplikace umí připojit k softwaru Syllabus Plus na přímo.
%To mi umožnilo zjednodušit celý proces o jeden krok a zároveň jsem díky tomu získal více možností (například naplánování časového lístku).
%
%V implementaci jsem pak dal návrhy dohromady a za použití několika knihoven jsem sestavil grafické rozhraní a plně funkční aplikaci.
%Bohužel, během této cesty jsem narazil na několik problémů.
%Většinu se mi podařilo vyřešit, ale dva v aplikaci zůstaly.
%Naštěstí jeden bude pravděpodobně opraven s aktualizací knihovny pro vykreslování uživatelského rozhraní a druhý se téměř neprojevuje a případně nemá zásadní dopad na funkčnost aplikace.
%
%V rámci testování jsem sestavil scénáře, které pokrývají veškeré funkcionality a slouží pro otestování samotné aplikace.
%Dále jsem provedl uživatelské testování s fakultním rozvrhářem, které odhalilo několik drobnějších chyb.
%Jejich vyřešením jsem odstranil překážky v intuitivnosti.
%
%Vytvořená aplikace splňuje všechny požadavky, řeší problémy aktuálního řešení a rozšiřuje možnosti při tvorbě rozvrhu.


\section{Possible Future Development}
%V rámci dalšího vývoje by se dala aplikace rozšířit o dvě nové sekce.
%Jednou jsou studijní plány a druhou kruhy studentů prvních ročníků.
%Tyto informace jsou aktuálně manuálně definovány fakultním rozvrhářem v softwaru Syllabus Plus.
%Díky návrhu aplikace by mělo být možné případnou funkcionalitu vytvořit přidáním pohledů a zpracováním nových dat.
%
%Dalším vylepšením by mohlo být převedení studentů a jejich zapsaných předmětů v daném semestru.
%Fakultní rozvrhář by tak mohl zahrnout do tvorby rozvrhu nové informace, které by mohly mít za důsledek zlepšení výsledné podoby rozvrhu.
%Například by se v případě vzniklých kolizí více časových lístků dalo zredukovat množství situací, kdy si zapsaní studenti nemohou sestavit nekolizní rozvrh.
%
%Na závěr dodám nápad na drobné vylepšení ovládání samotné aplikace, který byl vznesen během uživatelského testování.
%Jedná se o komplexnější práci s jednotlivými sloupci, kdy by mohlo být možné zobrazit například časové lístky pro více předmětů najednou.