%Na Fakultě informačních technologií (FIT) je potřeba každý semestr vytvořit nový rozvrh.
%Na to je nutné zjistit například, jaké předměty budou vyučované, kolik paralelek je potřeba vypsat a kdo bude tyto předměty učit.
%Většina informací je zadávána do Studijního informačního systému KOS\@.
%Pro následné vytvoření rozvrhu je používána rozvrhovací aplikace Syllabus Plus.
%Je proto potřeba umožnit převádět data mezi těmito dvěma systémy.
%
%Aby fakultní rozvrhář nemusel vše dělat ručně, existuje aktuálně řešení, které umí data jednorázově převést.
%Bohužel není dostatečně flexibilní a neumožňuje pokrýt velkou část požadavků.
%Kromě toho současné technické řešení není uživatelsky přívětivé, tedy rozvrhář neví, jaké konkrétní změny nastanou a jestli se mu neztratí již provedené úpravy.
%I proto řešení vyžaduje asistenci pro samotný běh.
%
%Toto téma jsem si zvolil, protože si myslím, že jsem schopný nabídnout řešení, které odstraní hlavní problémy a celkově usnadní tvorbu rozvrhu.
%Nový program tak ušetří čas a starosti fakultnímu rozvrháři.
%
%Práce je rozdělena na čtyři hlavní části, začíná analýzou, pokračuje návrhem a realizací a končí testováním.
%
%V první části se zabývám analýzou procesu tvorby rozvrhu a poté sběrem problémů a případných požadavků od fakultního rozvrháře.
%Na základě získaných informací sestavím výsledné požadavky na aplikaci a případy užití.
%
%V kapitole návrhu představuji nové řešení pomocí grafické aplikace, která dokáže data, ať už kompletní stav nebo jejich část, opakovaně převádět.
%Pro pohodlí uživatele, jednodušší převody a celkovou průhlednost fungování umí i přehledně vizualizovat data z jednotlivých systémů.
%Uživatel si tak dokáže lépe představit, která data se kde nachází a jaký převod potřebuje provést.
%Dále v této části popisuji vliv nového řešení na proces tvorby rozvrhu, strukturu dat systému KOS a softwaru Syllabus Plus a rozebírám možnosti práce s jednotlivými systémy.
%
%V kapitole realizace se zabývám výběrem vhodné technologie, implementací navrženého řešení a popisu architektury.
%Zabývám se zde i vzniklými problémy během implementace a jejich řešením.
%Kapitola končí rozborem použitých licencí jednotlivých knihoven.
%
%Testování řeším pomocí manuálních scénářů, které pokrývají požadavky aplikace a scénáře použití.
%Pro ověření pohodlnosti a efektivnosti práce s výslednou aplikací provádím uživatelské testování, které se zaměřuje na právě tento požadavek.